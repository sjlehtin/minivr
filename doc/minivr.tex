\documentclass[a4paper,twoside,titlepage,12pt]{article}
%
% Suomalaiset asetukset
\usepackage[finnish]{babel}
\usepackage[utf8]{inputenc}
\parindent0.0cm
\parskip0.23cm % Toista \tableofcontents:n jälkeen, muuten tulee harva sisällysluettelo
\frenchspacing
%
% Muita paketteja
\usepackage{listings}
\lstloadlanguages{SQL,Python}
\renewcommand{\lstlistingname}{Listaus}
\renewcommand{\lstlistlistingname}{Listaukset}

\usepackage{tikz}
\usepackage{graphicx}

% European Computer Modern fonts, scalable Type1 instead of bitmap fonts
\usepackage{ae}

%\pagestyle{headings}

%
% Hyperref pitää ladata viimeisenä
\usepackage{hyperref}

\begin{document}


%
% Etusivu
%
\begin{titlepage}

\vskip5.0cm

\begin{center}

{\huge
\vspace*{0.0cm}
\parskip1.0cm
\emph{Minivr}: junayhtiön tietojärjestelmä

Kurssin T-76.1143 harjoitustyö

{ \Large 5.11.2010 }

{ \Large Ryhmä 55: }

\large
Mikko Markus Torni $<$mtorni@cc.hut.fi$>$ (51161R) \\
    Sami J. Lehtinen $<$sjl@iki.fi$>$ (44814P) \\
    Matti Niemenmaa $<$mniemenm@cc.hut.fi$>$ (77243K)

}

\vskip5.0em

\vfill

\begin{flushleft}

\emph{Lähdekoodi:}

\indent \url{https://github.com/sjlehtin/minivr/}

\emph{Käyttöliittymädemo:} (tullaan vielä hiomaan)\\
\url{http://semeai.org:8000/}

\emph{Toteutusalusta:} Python + Django, PostgreSQL 8.0+

\emph{Demoajankohta:} 11.11. klo 15:20

\end{flushleft}

\end{center}


\end{titlepage}

\newpage

%
% Ensimmäinen sivu, sisällysluettelo
%

\parskip0.00cm % Ilman tätä tulee harva sisällysluettelo

\tableofcontents
\listoffigures
\lstlistoflistings

\parskip0.23cm % Vasta \tableofcontents:n jälkeen, muuten tulee harva sisällysluettelo

\newpage

%
% Ensimmäinen varsinainen tekstisivu
%

\section{Johdanto}

Teimme \href{http://www.aalto.fi/}{Aalto-yliopiston} \href{http://www.tkk.fi/}{Teknilliseen korkeakoulun} tiedonhallintajärjestelmät-kurssille (T-76.1143) vuoden 2010 syyslukukaudella harjoitustyön.



\subsection{Tehtävänanto: junayhtiön tietojärjestelmä}

Valitsimme kurssin T-76.1143 annetuista harjoitustyöaiheista junayhtiön tietojärjestelmän. Tehtävä oli annettu seuraavasti:

\begin{quotation}
\itshape
Asiakkaat voivat hakea palvelun kautta juna-aikatauluja, ja varata lippuja (jotka ovat haettavissa asemalta lähdettäessä), mikäli niitä on jäljellä. Lippujen hinnat riippuvat pääteasemista, junavuorosta sekä ostajan statuksesta (opiskelija/aikuinen jne.). Eri junavuoroilla on omat lähtö -ja päätepysäkit, eivätkä junavuorot välttämättä pysähdy kaikilla asemilla. Reitit haarautuvat rautatieverkostossa täysin mielivaltaisesti. Esimerkiksi pysäkiltä B voi olla kiskot pysäkeille A, C, D ja E, eikä ainoastaan ``edelliselle pysäkille A ja seuraavalle pysäkille C''. Järjestelmän tulisi pystyä näyttämään käyttälle reittioppaan tavoin eri vaihtoehdot matkustaa paikasta x paikkaan y, huomioiden mahdolliset vaihdot, sekä optimoiden ajankäytön ja hinnan.

Junavuoron voi olettaa kulkevan eri kausina (kesä/talvi) tietyn viikonpäivän tiettyyn kellonaikaan. Eri junavuorojen nopeuksista toisiinsa nähden ei voi tehdä mitään oletuksia.

Aihe tarjoaa paljon haastetta ja päänvaivaa erityisesti tietokannan rakenteen suunnittelussa. Haastetta voi kasvattaa entisestään esimerkiksi muuttamalla osan junavuoroista lähijuniksi, jolloin lippuja ei voi varata ennalta, ja hinnoittelu tapahtuu kuljettujen vyöhykkeiden perusteella. Toisaalta kaukojunien liput voisi pystyä varaamaan tietylle istumapaikalle tiettyyn vaunuluokkaan (travel/business).

Aiheeseen voi ottaa ideoita esim. VR:n sivuita.
\end{quotation}

\subsection{Järjestelmän vaatimukset}
Käyttäjät etsivät järjestelmää käyttäen itselleen sopivia junavuoroja.
Järjestelmän pitää pystyä hakemaan käyttäjälle reitti annettujen asemien
välille. Jos suoraa yhteyttä ei löydy, järjestelmä yrittää löytää käyttäjälle
mahdollisimman hyvän vaihtoyhteyden. Käyttäjän pitää pystyä varaamaan lippuja
valitsemalleen reitille.

\lstset{language=SQL}

\section{Tietokanta}

Valitsimme tietokannaksi vapaan PostgreSQL\footnote{\url{http://www.postgresql.org/}}-tietokannan. PostgreSQL on ohjelmistosuunnittelijaystävällinen sekä tarjoaa suorituskykyä ja joustavan indeksoinnin.

\subsection{ER-malli}

\begin{figure}
  \includegraphics[scale=0.5,angle=270]{relations}
  \caption{ER-malli keskeisimmista relaatioista}
\end{figure}

Alla lyhyt selitys kunkin relaation merkityksestä.

\subsubsection{CustomerType}

Asiakastyyppi lipulle. Pääasiassa alennuslippujen määrittelyyn. Sisältää vain
käyttäjäystävällisen nimen, jotta sitä ei tarvitse toistaa jokaisessa
lipputyypissä.

\begin{lstlisting}
INSERT INTO minivr_customertype
            (id, name)
     VALUES (1, 'Aikuinen'),
            (2, 'Lapsi')
\end{lstlisting}

\subsubsection{Ticket}

Vuorolle varattu lippu. Sisältää \emph{Connection.cost}-kentästä riippuvan
hinnan sekä viittaukset junavuoroon ja asiakastyyppiin.

Erillistä \emph{id}-kenttää ei periaatteessa tarvita, mutta käyttämämme
ohjelmisto ei tue monisarakkeisia pääavaimia: kenttä on vain myönnytys sille.

\begin{lstlisting}
INSERT INTO minivr_ticket
            (id, service_id, customer_type_id,
             price_per_cost)
     VALUES (0, 0, 0, 0.859600),
            (1, 0, 1, 0.331600),
            (2, 2, 1, 1.613300)
\end{lstlisting}

\subsubsection{Station}

Juna-asema. Kuten \emph{CustomerType}, tämä relaatio sisältää vain nimen.

\begin{lstlisting}
INSERT INTO minivr_station
            (id, name)
     VALUES (0, 'Helsinki'),
            (1, 'Pasila'),
            (2, 'Tikkurila')
\end{lstlisting}

\subsubsection{Train}

Juna, jolla vuoro ajetaan. Nimen ohella relaatiossa on tieto siitä, miten monta
asiakaspaikkaa junassa on.

\begin{lstlisting}
INSERT INTO minivr_train
            (id, name, seats)
     VALUES (0, 'IC2 109', 120),
            (1, 'IC 11', 120),
            (2, 'IC 113', 84)
\end{lstlisting}

\subsubsection{Service}

Junavuoro, esimerkiksi \emph{IC2 109 klo 12:15}, ja kuinka monta tyhjää paikkaa
siinä on jäljellä. Oletusarvoisesti kukin juna kulkee joka päivä.

Sovellus tukee varauksia vain yhteen vuoroon kerrallaan eli esim. päivittäin
kulkevaan junaan ei voi varata paikkaa ylihuomiselle. Tämä rajoitus tajuttiin
liian myöhään, joten se päätettiin jättää sikseen. (Eräs mahdollinen ratkaisu
esitelty \emph{Stop}-relaation yhteydessä jäljempänä.)

\begin{lstlisting}
INSERT INTO minivr_service
            (id, train_id, departure_time, free_seats)
     VALUES (0, 60, '12:15', 26),
            (1, 90, '13:50', 97)
\end{lstlisting}

\subsubsection{Connection}

Kuvaa raideyhteyttä kahden aseman välillä. Relaatio on vain maantieteellinen
eikä ota kantaa esimerkiksi siihen, miten kauan kestää kulkea kyseinen väli
tietyllä junavuorolla.

Yhteydessä on kaksi eri mittaa: \emph{distance} on etäisyys, \emph{cost} taas
abstraktisti ``kulkuhintaa'' merkitsevä. Se voi esimerkiksi olla etäisyyttä
korkeampi, jos rata on tavallista kalliimpi ylläpitää.
\emph{Ticket}-relaatiossa mainittu hinta on yhtä \emph{cost}-yksikköä kohden.

Kuten \emph{Ticket}:n tapauksessa, \emph{id}-kenttä on myönnytys ohjelmistolle.

\begin{lstlisting}
INSERT INTO minivr_customertype
            (id, out_of_id, to_id, distance, cost)
     VALUES (0, 0, 1, 32, 26),
            (1, 1, 0, 32, 26),
            (2, 1, 2, 44, 39),
            (4, 2, 1, 44, 39)
\end{lstlisting}

\subsubsection{Stop}

Yksittäinen pysähdys asemalla junavuorolle. Sisältää tiedon siitä, milloin
vuoro saapuu kyseiselle asemalle ja milloin se lähtee sieltä.

\emph{arrival\_time}- ja \emph{departure\_time}-kentät ovat minuuteissa alkaen
vuoron lähtöajasta---tämä esitystapa valittiin, koska se on kätevä reitinhaun
toteutuksen kannalta. Vuoron alkuasemalla \emph{arrival\_time} ja vastaavasti
pääteasemalla \emph{departure\_time} on tyhjä kenttä (NULL).

Rivissä on myös tieto siitä, milloin kyseinen pysähdyspaikka on voimassa.
Allaolevassa esimerkissä vuoro pysähtyy asemilla vuosien 2010--2011 jokaisen
kuukauden jokaisena viikonpäivänä. Kuuden päivämääräkentän avulla voidaan
esittää kausittaisia pysähdyspaikkoja tyyliin ``tämä vuoro pysähtyy täällä vain
kesämaanantaisin''. Vuodet voivat olla NULL-arvoja, jolloin niitä ei oteta
huomioon.

Kuten \emph{Ticket}:n tapauksessa, \emph{id}-kenttä on myönnytys ohjelmistolle.

\begin{lstlisting}
INSERT INTO minivr_stop
            (id, service_id, station_id,
             arrival_time, departure_time,
             year_min, year_max, month_min, month_max,
             weekday_min, weekday_max)
     VALUES (0, 0,  25,NULL, 0,2010,2011,1,12,1,7),
            (1, 0,  28,  14,15,2010,2011,1,12,1,7)
\end{lstlisting}

\subsection{Tietokantatoteutus}

Taulujen luontikomennot ovat listauksessa~\ref{lst:sqlcreatetables} sivulla~\pageref{lst:sqlcreatetables}. Tietokantataulut luotiin Djangon objekti-relaatiokuvausmoduulia käyttäen, joka loi SQL-taulut ja päivitti tietokantarakenteet kehityksen aikana.

Suoraviivaisesta ER-mallin muunnoksesta relaatiomalliksi täytyi poiketa monissa kohdissa johtuen käytetyn objekti-relaatiokuvaustekniikan rajoituksista. Tekniikka ei sallinut muunmuassa useamman attribuutin avaimia. Tietokantaan ei voinut myöskään laittaa kuin yksinkertaisia tarkistuksia tiedon oikeellisuudesta.

Käytetyn tekniikan rajoituksia olisi voinut osin kiertää käyttämällä tietokantapalvelimeen tallennettuja proseduureja tai näkymiä. Projektin yhteydessä haluttiin kuitenkin opetella objekti-relaatiokuvauksen käyttöä sen ketteryyden ja käytännöllisyyden vuoksi.

\subsubsection{Testidata}

Pohjana testidatalle on ote VR:n junavuorojen aikatauluista CSV\footnote{Comma Separated Values}-muodossa. Lähdekoodin tiedostossa \texttt{schedules.csv} on riveittäin lueteltu junavuoron tunnus, pysäkki, saapumisaika asemalle ja lähtöaika asemalta.

\begin{figure}
\begin{lstlisting}
IC2 109,Helsinki,,14:12
IC2 109,Pasila,14:17,14:18
IC2 109,Tikkurila,14:27,14:28
IC2 109,Lahti,15:06,15:08
IC2 109,Kouvola,15:40,
IC 11,Helsinki,,18:12
\end{lstlisting}
[\ldots]
  \caption{Ote junavuorojen testidatasta}
\end{figure}

Projektissa testidata generoidaan tietokantaan Python-skriptillä, joka luo tarvittavat tietorakenteet ja täydentää syöttötietoja satunnaisdatalla muunmuassa etäisyydet ja hinnat.

\subsubsection{Normalisointi}
\textbf{TODO:} Esitys funktionaalisista riippuvuuksista taulujen attribuuttien välillä. Ovatko taulut BCNF-normaalimuodossa? BCNF-normaalimuodon rikkominen edellyttää perustelut.

\subsubsection{Indeksointi}
Junayhteyksien haku on toteutettu Dijkstran algoritmilla sovelluksessa siten että tietokannalle tehdään tiheästi yksinkertaisia kyselyitä ja datamäärä on pieni. Indekseistä on rajoitetusti apua vain suuremmissa tauluissa joissa on yli tuhat riviä testidataa (\emph{Connection}, \emph{Stop}).

Tauluissa on indeksejä primääriavaimien lisäksi seuraavasti:

\begin{tabular}{ll}
\emph{Ticket} & service\_id, customer\_type\_id\\
\emph{Service} & train\_id\\
\emph{Stop} & service\_id, station\_id\\
\emph{Connection} & out\_of\_id, to\_id\\
\end{tabular}

\subsection{Tärkeimmät tietokantakyselyt}

\subsubsection{Junayhteyksien haku Dijkstran algoritmilla}
Toteutetun työn tärkein osuus on tarjota mahdollisia junayhteyksiä asiakkaalle. Tietokantaan on tallennettu saapumis- ja lähtöaikoja junavuoroittain. Ohjelman tehtävänä on päätellä kelvolliset suorat ja vaihdolliset junayhteydet ja esittää ne yksinkertaisessa muodossa.

Yksi mahdollinen tapa ratkaista ongelma on ajatella toteutuvia junavuoroja suunnattuna verkkona (jossa voi olla silmukoita). Ratkaisun voi löytää soveltamalla jotain tunnettua algoritmia polkujen löytämiseksi.

Toteutimme reittihakuihin muunnelman Dijkstran algoritmista\footnote{Katso algoritmin kuvaus esimerkiksi \href{http://fi.wikipedia/Dijkstran\_algoritmi}{Wikipediasta}} joka löytää vaihtoehtoiset reitit.

\lstset{language=SQL,caption=Esimerkki kyselylauseesta}
\begin{lstlisting}
SELECT * FROM
  (SELECT
     (60 * extract(hour   from minivr_service.departure_time)
       +   extract(minute from minivr_service.departure_time)
       + minivr_stop.departure_time
       - ?)
     AS t,
     minivr_stop.*
     FROM minivr_stop
         INNER JOIN minivr_service
                 ON (minivr_stop.service_id = minivr_service.id)
         INNER JOIN minivr_station
                 ON (minivr_stop.station_id = minivr_station.id)
     WHERE UPPER(minivr_station.name::text) = UPPER(?)
       AND minivr_service.free_seats > 0
       AND (minivr_stop.year_min IS NULL
            OR minivr_stop.year_min <= wanted_year)
       AND (minivr_stop.year_max IS NULL
            OR minivr_stop.year_max >= wanted_year)
       AND minivr_stop.month_min <= wanted_month
       AND minivr_stop.month_max >= wanted_month
       AND minivr_stop.weekday_min <= wanted_weekday
       AND minivr_stop.weekday_max >= wanted_weekday
       AND minivr_stop.departure_time IS NOT NULL)
  AS ts
  ORDER BY ts.t - (24*60) * floor(ts.t / (24*60)) DESC)
\end{lstlisting}

\section{Käyttöliittymä}

Ryhmän jäsenten kiinnostuksen perusteella valitsimme käyttöliittymän toteutukseen \href{http://www.python.org/}{Python}-ohjelmointikielen\footnote{\url{http://www.python.org/}} ja Python-pohjaisen \href{http://www.djangoproject.com/}{Django-kehitysrungon}\footnote{\url{http://www.djangoproject.com/}}.

Django on korkean tason Web-kehitysympäristö joka kannustaa nopeaan kehitykseen ja puhtaaseen, pragmaattiseen suunnitteluun. Djangon ominaisuuksista käytimme objekti-relaatiokuvausta, jonka avulla ohjelmoidessa ei tarvitse erikseen työstää SQL-rajapinnan tietotyyppimuunnoksia.

Ohjelmistokehityksen apuna käytimme Djangon mukana tulevaa kehityswebbipalvelinta joka soveltuu hyvin prototyyppivaiheeseen.

Tässä osassa esittelemme käyttöliittymän keskeisimmät sivut ja toiminnallisuudet.

\subsection{Yleistä}
Käyttäjä voi useimmissa ruuduissa klikata junavuoron nimeä jolloin näkyviin tulee asemakohtaiset aikataulut, tai käyttää \emph{Varaa}-painiketta paikan varaamiseen junavuoroon.

\subsection{Reitin haku}
Sivulla käyttäjä voi hakea vuoroja asemalta mille tahansa muulle asemalle. Käyttäjän täytyy syöttää toivottu lähtö- tai saapumisaika. Haun tuloksena käyttäjälle näytetään kaikki mahdolliset junavuorot saman vuorokauden sisällä.

Haun suorittamisen jälkeen käyttäjällä on mahdollisuus varata paikka haluamaansa junavuoroon painamalla \emph{Varaa}-painiketta. Järjestelmä ilmoittaa onnistuiko varaus.

\begin{figure}
  \includegraphics[width=130mm]{route.png}
  \caption{Reitin haku}
\end{figure}

\subsection{Kaikki asemat}
Sivulla näkyy lista kaikista asemista.

\subsection{Varauksen onnistuminen}

Reittihausta varauksen onnistuminen (tai epäonnistuminen) näytetään
käyttäjälle erillisellä sivulla.

Tältä sivulta pitäisi pystyä palaamaan rettihakusivulle (tämä on vielä
tekemättä).

\section{Viimeistely}

Ryhmällä on tarkoitus vielä lisätä lisätä linkkejä sivujen välille
käyttökokemuksen parantamiseksi (mm. varauksen vahvistussivulta takaisin
reittihakuun).

Virhesivuja ei ole viimeistelty (kaksi yhtäaikaista varausta viimeiseen
paikkaan, toinen siis epäonnistuu).

Reitinhaku sallii vaihdon junavuorosta toiseen kesken matkan, jos saapumis- ja
lähtöaikojen ero ylittää kovakoodatun rajan: 5 minuuttia. Demoon mennessä on
tarkoitus antaa käyttäjälle mahdollisuus valita tämä raja.

\section{Loppusanat}
\subsection{Työmäärä}
\textbf{TODO:} Matti Niemenmaan työtunnit

Mikko Markus Torni käytti harjoitustyöhön aikaa yhden tunnin ideoiden ylös kirjaamiseen, 12 tuntia dokumentoinnissa ja kaksi tuntia koodiin tutustumiseen.

Sami J. Lehtinen käytti 4 tuntia ylläpitotoimiin, 20 tuntia koodaamiseen
ja Djangoon tutustumiseen sekä 4 tuntia dokumentointiin.

\subsection{Oma arvio harjoitustyöstä}

Ryhmä aloitti työn tekemisen toden teolla aivan liian myöhään.

\subsection{Palaute kurssin järjestäjille}

Ei valittamista kurssijärjestelyistä.

%
% Liitteet
%
%
\newpage
\section*{Tietokannan luontikomennot}

\lstinputlisting[language=SQL,breaklines=true,breakatwhitespace=true,frame=single,basicstyle=\small,label=lst:sqlcreatetables,caption=Tietokantataulujen luontikomennot]{minivr.sql}

\end{document}
